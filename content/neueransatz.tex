
Ich gehe von der Idee aus, dass das System gegen ein Quasi-Gleichgewicht (QGGW) strebt. 
Inspiriert vom BM-Gleichgewicht, gehe ich davon aus, dass das QGGW durch ein konstantes Verhaeltnis von $n_1$ zu $n_0$ charakterisiert wird und bei Stoerung das QGGW verschoben ist. 
Diese Idee hatte ich schon in meiner Dissertation geaeu3ert, wenn auch nicht sauber mathematisch ausformuliert. 
Die Idee selbst greifen Sie auch in Ihrem Dokument auf, und argumentieren, dass dies fuer kleine Stoerungen der Fall sei (section 4.3.2 High-temperature case). 
Fuer diese Limitierung sehe ich keinen Grund. 
Ich gehe also einen Schritt weiter und sage: 
Sobald es in einem System von zwei Niveaus mindestens eine Zeitkonste ``schnell'' ist, also schneller als die experimentelle Aufloesung, 
 dann befindet sich das System hinreichend schnell in einem QGGW-Zustand. 
Denn die Stoerungen durch Drift koennen zwar gro3 sein (schnell), aber sie sind zeitlich konstant ($\taudrift, \taurec$ sind keine Funktion der Zeit),
 und somit stellt sich ein QGGW mit der schnellsten Zeitkonstante ein.
 
Beispiel bei 500\,V, 100\,K:
$\taudrift$ ist bei 500V klein ($\approx$50\,ps), $\tauevap$ ist ca 14\,ns, $\taurec$ ca 14\,ns, $\taub$ ist anfangs immer schnell, da die Dichte anfangs hoch ist.
Also wird $n_1$ schnell durch $\taub$ oder ``$\taudrift$ und $\taub$'' entleert. 
$n_0$ fuellt sich also schnell, aendert sich aber nicht stark da $\taurec$ und $\tauevap$ langsam sind. 
Es stellt sich ein QGGW ein, wo das Verhaeltnis $n_1/n_0$ entsprechend klein ist, da $\taudrift$ viel schnell ist als $\taurec$.
Falls nun $\tauevap$ auch schnell sein sollte, verschiebt sich nur das QGGW, aber es wird weiterhin schnell erreicht. 
So kann man nun beliebige Kombinationen durchspielen.
Solange mindestens eine Zeitkonstante schnell ist, strebt das System meiner Meinung nach hinreichend schnell gegen einen QGGW-Zustand, da fuer das schnelle Erreichen dieses Zustandes, 
 sich nur eine Niveau schnell aendern muss. 
 
Man gehe von Ihrem DGL-System aus

\begin{align}
\frac{\dd n_0}{\dd t} &= -\frac{n_0}{\tauzero} + \frac{n_1}{\taub} \qquad \textrm{with}    \qquad\frac{1}{\tauzero}= \frac{1}{\tauevap} + \frac{1}{\taurec} \\
 \frac{\dd n_1}{\dd t} &= +\frac{n_0}{\tauevap} - \frac{n_1}{\tauone}  \qquad \textrm{with} \qquad\frac{1}{\tauone} = \frac{1}{\taub} + \frac{1}{\taudrift} 
 \label{math:diff22}
\end{align}

\noindent
Au3erdem beachte man, dass $\taub = \frac{1}{\sigma v n_1}$. 
Dieses $n_1$ kann nur die Dichte der freien Loecher sein, mit denen ein freies Elektron ein Exciton bilden kann. 
Das beudetet aber, dass $\taub$ per se nicht zeitlich konstant ist, sonder anfangs schnell (es gibt noch viele freie Loecher, Dichte 1e18 / cm3), und spaeter unter Umstaenden viel langsamer (bei Dichte 1e15 1/cm3), 
 mit entsprechenden Konsequenzen fuer das Modell. 
%Man beachte, es gibt nicht ein $\taub$ und ein anderes $t_0$, sondern nur $\taub$ wie oben geschrieben. 
Zusaetzlich muss auch $\tauevap$ ueberdacht werden. 
In Ihrem Dokument zur Vorbereitung der Publikation ist $\tauevap = t_0 *\exp(...)$, mit $t_0 = \frac{1}{\sigma v n_1}$, wobei $n = n_1 = $ ca 1e18 angenommen wurde. 
Ich halte dagegen, dass $n$ hier nicht die Dichte der freien Elektron/Loecher sein kann, sondern die Zustandsdichte im Leitungs-/Valenzband sein muss,
 denn wenn ein Exciton thermisch ionisiert, muessen genau dort die Ladungen einen Platz finden. 
Es gilt also nicht $\tauevap = \taub*\exppEx$. 
Ich behaupte also $\tauevap = \frac{1}{\sigma v N_V} \exppEx \equiv t_{e,0}\exppEx$, welches nachher ueberprueft werden soll, siehe weiter unten. 

ANNAHME:\\
Ein Quasi-GGW wird schnell, innerhalb der experimentellen Zeitaufloesung (ca 360\,ps), erreicht. 
Man denke an $\taudrift$ im Bereich von 50\,ps, welches das QGGW schnell herstellt. 
In diesem QGGW ist nicht etwa $\dd n / \dd t = 0$, sondern das Verhaeltnis $n_1/n_0$ ist konstant.
Damit folgt

\begin{equation}
 \frac{\dd}{\dd t} \left( \frac{n_1}{n_0}\right) = 0 \Rightarrow  \frac{n_1}{n_0} = f \quad \textrm{oder} \quad n_1 = f\cdot n_0,\,\,f>0.
 \label{eq:f}
\end{equation}

\noindent
Setzt man eq.~(\ref{eq:f}) in eq.~(37) ein, erhaelt man:

\begin{equation}
 \frac{\dd n_0}{\dd t} =  -\frac{n_0}{\tauzero} + f\cdot \frac{n_0}{\taub} \equiv -\frac{n_0}{\tauzeroprime}, \qquad \frac{1}{\tauzeroprime} = \frac{1}{\tauzero} - \frac{f}{\taub}.
\end{equation}
\noindent
Es folgt:

\begin{equation}
 n_0 = n_0 (0) \exp(-t / \tauzeroprime).
 \label{eq:n0}
\end{equation}
\noindent
$n_1(t)$ folgt direkt aus aus eq. (39). 
Benutze (38) zur Bestimmung von $f$ (ohne weitere Annahmen):

\begin{equation}
 \frac{\dd n_1}{\dd t} = f \frac{\dd n_0}{\dd t} \stackrel{!}{=} \frac{n_0}{\tauevap} - \frac{n_1}{\tauone}
\end{equation}
\noindent
Nach Einsetzen von eq.(41), Ableiten und Aufloesen nach $f$ folgt:

\begin{equation}
 \boxed{\frac{f^2}{\taub} - \frac{f}{\tauzero} = \frac{1}{\tauevap} - f \left( \frac{1}{\taub} + \frac{1}{\taudrift} \right)}.
 \label{eq:solvef}
\end{equation}
\noindent
Die einzige Annahme war, dass das System gegen ein Q-GGW strebt bei dem das Verhaeltnis der Dichten zeitlich konstant ist,
 und dieses Q-GGW innerhalb der experimentellen Aufloesung erreicht wird. 
Es musste keine Annahmen ueber die Groe3en der Zeitkonstanten oder Dichten getroffen werden. 
Eq.~(\ref{eq:solvef}) kann und wird nun fuer verschiedene Faelle ausgewertet. 

\subsection{Ohne Stoerung}

Ohne Stoerung ($\taudrift,\taurec \rightarrow \infty$) folgt aus Eq. (\ref{eq:solvef}):

\begin{equation}
 \frac{f^2}{\taub} + f\left( \frac{1}{\taub} - \frac{1}{\tauevap} \right) = \frac{1}{\tauevap} 
\end{equation}

Loesungen von eq (44) sind $f = -1$ (unphysikalisch) und $f = \frac{\taub}{\tauevap}$ (!), wie man schnell zeigen kann. Es folgt also im Fall ohne Stoerung:

\begin{equation}
 \frac{n_1}{n_0} = f = \frac{\taub}{\tauevap} = c \cdot \expmEx, \,\,\,\textrm{mit}\,\,\, c = \frac{N_V}{n_1}.
\end{equation}

\noindent
Das ist genau der BM-Faktor mit dem Verhaeltnis der Zustandsdichten $c$. 
N.B.: Je nachdem wie gro3 $N_V$ ist, kann $f$ sogar auch $>1$ sein.
In der Literatur finde ich fuer Diamant $N_V \approx 1e19$\,cm-3.

\subsection{Mit gleicher Stoerung}
Fuer Stoerungen gleicher Groe3e, $\taudrift = \taurec$, reduziert sich Eq. (\ref{eq:solvef}) wie im Fall ohne Stoerung.

\begin{equation}
 \frac{f^2}{\taub} + f\left( \frac{1}{\taub} - \frac{1}{\tauevap} \right) = \frac{1}{\tauevap} \Rightarrow f = \frac{\taub}{\tauevap}. 
\end{equation}

\noindent
Es folgt wieder das QGGW mit BM-Faktor. 

\subsection{Allgemeine Loesung}
Die allgemeine Loesung lautet (ich schreibe nun erst $\tauzero$ aus):

\begin{equation}
  \frac{f^2}{\taub} + f \left( \frac{1}{\taub} + \frac{1}{\taudrift} - \frac{1}{\tauevap} -\frac{1}{\taurec} \right) = \frac{1}{\tauevap}.
 \label{eq:solvefallgemein}
\end{equation}

\noindent
Ich vereinfache zu $\frac{f^2}{\taub} + \frac{f}{\tauoneprime} = \frac{1}{\tauevap}$ mit $\frac{1}{\tauzeroprime} = \frac{1}{\taub} + \frac{1}{\taudrift} - \frac{1}{\tauevap} -\frac{1}{\taurec} $ und den Loesungen:

\begin{equation}
 f_{1/2} = \pm \sqrt{\frac{\taub}{\tauevap} + \left( \frac{\taub}{2\tauoneprime}\right)^2 } - \frac{\taub}{2\tauoneprime}. 
\end{equation}

\noindent
$f_2$ (negative Loesung) ist wieder unphysikalisch. 
Leider laesst sich mit $f_1$ aber nicht gut arbeiten, da ja die gemessene Zeitkostante $\frac{1}{\tauzeroprime} = \frac{1}{\tauzero} - \frac{f}{\taub}$ ist, und $f_1$ zu sperrig ist. 


\subsection{Allgemeine Loesung fuer kleine Stoerungen}
\noindent
Die Wurzel in (48) laesst sich analytisch nicht aufloesen. 
Fuer kleine Stoerungen im Vergleich zum BM-Verhaeltnis ersetze ich $f = \frac{\taub}{\tauevap}(1 + \delta)$. 
Eingesetzt in (47) und ich vernachlaessige nur den $\delta^2$-Term. 
Die allgemeine Loesungen lautet dann 
\begin{equation}
 f = \frac{\taub}{\tauevap}(1 + \delta)\,\,\,\textrm{mit} \,\,\, \delta = \frac{\frac{1}{\taurec} - \frac{1}{\taudrift}}{\frac{1}{\tauevap} + \frac{1}{\taub} + \frac{1}{\taudrift} - \frac{1}{\taurec}}. 
\end{equation}

\noindent
Fuer den fall ohne (gleicher Stoerung) erhaelt man $\delta = 0$ und somit das BM-Verhaeltnis. 
Mit Stoerung spiegelt der Zaehler das Vorzeichen der Veraenderung wider: 
Wenn der Drift schneller ist als die Rekombination, ist $\delta$ negativ, und $f$ verschiebt sich zu kleiner Werten, wie man es auch erwarten wuerde.
Wenn die Rekombination schneller ist als der Drift, ist $\delta$ positiv, und $f$ verschiebt sich zu groe3eren Werten.
Die gemessene Zeitkonstante ist dann $\frac{1}{\tauzeroprime} = \frac{1}{\tauzero} - \frac{f}{\taub} = \frac{1}{\taurec} - \frac{\delta}{\tauevap}$.

\subsection{Vereinfachte allgemeine Loesung beliebiger Stoerung}
Nun ist die Frage, wie gro3 $f$ in unserem Experiment ist, also mit $\taudrift$ im Bereich (grob) 50 bis 1000\,ps und $\taurec$ ca 10\,ns. 
Das haengt natuerlich von $\taub$ ab, welches aber wiederum von $n_1$ abhaengt.
Nimmt man also $\taub = \frac{1}{\sigma v n_1}$ an, muesste das DGL-System eigentlich so lauten:

\begin{align}
 \frac{\dd n_0}{\dd t}  &= -\frac{n_0}{\tauzero} + \sigma v n_1^2  \qquad \textrm{mit}    \qquad\frac{1}{\tauzero}= \frac{1}{\tauevap} + \frac{1}{\taurec} \\
 \frac{\dd n_1}{\dd t}  &= +\frac{n_0}{\tauevap} - \sigma v n_1^2 - \frac{n_1}{\taudrift}
 \label{math:diff33} 
\end{align}



\noindent
Nun habe ich fuer diese quadratische System keine Loesung parat, deswegen hier eher vollstaendigkeits-halber angegeben. 
Aber z.B.\ bei kalten Temperaturen ($\tauevap$ gro3) und ohne Drift ($\taudrift \rightarrow \infty$) reduziert sich (51) zu:

\begin{equation}
 \frac{\dd n_1}{\dd t} = - \sigma v n_1^2.
\end{equation}

\noindent
Dann ist eine Loesung:

\begin{equation}
 n_1 = \frac{1}{\sigma v t + 1/n_1(0)},
\end{equation}

\noindent
und entsprechend nimmt $n_1$ anfangs sehr schnell ab (in 180\,ps auf ca 1.6e16 1/cm3). 
Danach sollte (dann mit Drift), die Aenderung durch den Drift dominant werden, und $n_1$ nimmt exponentiell mit $\taudrift$ ab, wie gesagt, die Ueberlegung gilt fuer gro3e $\tauevap$. 
Nach $3*\taudrift$ ist $n_1$ ca 8e14 1/cm3, und $\boxed{\taub = \SI{3}{\nano\second}}$!

Diese Ueberlegung soll verdeutlichen, dass $\taub$ eigentlich zeitabhaengig ist, und wie diese Abhaenigkeit ungefaehr aussieht. 
Was waere der Wert von $f$ fuer solch einen Fall, wenn also $\taub \approx \SI{3}{\nano\second}$ und sagen wir $T = \SI{100}{\kelvin}$:

\begin{align}
  \frac{f^2}{\taub} + f \left( \frac{1}{\taub} + \frac{1}{\taudrift} - \frac{1}{\tauevap} -\frac{1}{\taurec} \right) &= \frac{1}{\tauevap}\\
  \Rightarrow
  \frac{f^2}{3} + f \left( \frac{1}{3} + \frac{1}{0.06} - \frac{1}{14} -\frac{1}{14} \right) &= \frac{1}{14}\\
  \Rightarrow 
  f &= 4.3e-3.
\end{align}

\noindent
Der Zahlenwert ist nicht von Bedeutung, aber das Faktum, dass $f$ klein ist. 
Eine aehnliche Rechnung fuer T = 150\,K kommt auf $f \approx 0.2$.

Starten wir nun mit der Ausgangsgleichung von $f$:
\begin{equation*}
 \frac{f^2}{\taub} - \frac{f}{\tauzero} = \frac{1}{\tauevap} - f \left( \frac{1}{\taub} + \frac{1}{\taudrift} \right).
\end{equation*}

\noindent
Wenn $f$ klein ist, und $\taub \approx \SI{3}{\ns}$ ist $\frac{f^2}{\taub} \ll \frac{f}{\tauzero}$, und $\taub \gg \taudrift$ und eben nicht kleiner! %siehe dazu auch Anmerkung A??). 
Somit vereinfacht sich die Gleichung zu:

\begin{equation}
  f \left( \frac{1}{\tauevap} + \frac{1}{\taurec} \right)= -\frac{1}{\tauevap} + \frac{f}{\taudrift}
\end{equation}

\noindent
und man erhaelt

\begin{equation}
 \boxed{ f = \frac{\frac{1}{\tauevap}}{\frac{1}{\taudrift} - \frac{1}{\tauevap} - \frac{1}{\taurec}} }. 
\end{equation}

\subsubsection{Fuer $\taudrift \ll \tauevap$}
Wenn nun $\taudrift \ll \tauevap$ (bis ca.\ $T < 150$\,K: $\tauevap \approx \SI{400}{\ps}$ bei 150\,K und $\taudrift$ bei 500 V ca 50\,ps), gilt 
\begin{equation}
 \boxed{f = \frac{\taudrift}{\tauevap}}\,\,\, \textrm{und } \,\,\, \boxed{\frac{1}{\tauzeroprime} = \frac{1}{\tauzero} - \frac{f}{\taub} \approx \frac{1}{\tauzero} = \frac{1}{\tauevap} + \frac{1}{\taurec}}, 
\end{equation}
wie ``erhofft'', da damit $n_1$ in bekannter Form auftaucht:

\begin{equation}
 n_1 = \frac{\taudrift}{\tauevap} n_0(0) \exp{(-t / \tauzero)}. 
\end{equation}

\noindent
und $\ndrift$, mittels $\frac{\dd \ndrift}{\dd t} = + \frac{n_1}{\taudrift}$:

\begin{equation}
 \ndrift = \frac{\tauzero}{\tauevap} n_0(0) (1 - \exp{(-t / \tauzero)}).
\end{equation}

\noindent
Der Faktor $\frac{\tauzero}{\tauevap}$ kann wie immer umgeschrieben werden in $\frac{1}{1 + \tauevap/\taurec} = \frac{1}{1 + t_{e,0}\exppEx/\taurec}$.
Das ist genau der Faktor, den ich auch benutzt habe, um die Fits an $Q=Q(T)$ durchzufuehren. 
Zur Erinnerung, hier ist $t_{e,0} = \frac{1}{\sigma v N_V}$. 
Experimentell wurde bei T = 100\,K $\tauevap = \SI{14}{\ns}$ bestimmt. 
Nimmt man nun den spannungsabhaenigen Wert $\Ea(\SI{500}{\volt}) = \SI{0.09}{\eV}$ an, ist $t_{e,0} = \SI{0.3}{\ps}$, und mit Ihren Werten fuer $\sigma$ und der thermischen Geschwindigkeit erhaelte ich
$\boxed{N_V = \SI{1.4e19}{\centi\meter^{-3}}}$, in passabler Uebereinstimmung mit Literaturwerten. 
Fuer die innere Ladung finde ich:

\begin{equation}
 \Qin \propto n_1(0) t' + n_0(0) \frac{\tauzero}{\tauevap}\cdot \left[ t' - \tauzero(1-\exp{(-t'/\tauzero)} \right]
\end{equation}

\noindent
oder wenn bis $t\rightarrow \infty$ integriert wird:
\begin{equation}
 \Qin \propto n_1(0) t' + n_0(0) \frac{\tauzero}{\tauevap}\cdot \left[ t' \right]
\end{equation}

\noindent
Mit Faktoren
\begin{equation}
 \Qin = Q_1(0) + Q_0(0) \frac{\tauzero}{\tauevap}
\end{equation}

\noindent
Gesamtladung (inklusive $\Qo$) bis unendlich:
\begin{equation}
 \Qtot = \Qo(0) + Q_1(0) + Q_0(0) \frac{\tauzero}{\tauevap}
\end{equation}

\noindent
Wenn man nun annimmt, dass $Q_1(0)$ vernachlaessigbar ist und $Q_0(0) \approx \Qin$, folgt, wie frueher:

\begin{equation}
 \boxed{\Qtot = \Qo(0) + \Qin \frac{\tauzero}{\tauevap}}.
\end{equation}

\noindent
Bei 100\,K ist der Fehler im Promillbereich oder weniger, bei 150\,K macht man einen Fehler von ca 15\,\%, da $f(\SI{150}{\kelvin}) \approx 0.2$. 
Nimmt man die Veraenderung von $Q_1(0)$ als Funktion der Temperatur mit, ergibt sich mit $f = n_1/n_0 = Q_1 / Q_0$ und entsprechend $Q_1 = \frac{f}{1+f}\Qin$ und $Q_0 = \frac{1}{1+f}\Qin$:

\begin{equation}
 \Qtot = \Qo(0) + \Qin\cdot \left( \frac{f}{1+f} + \frac{\tauzero}{\tauevap}\frac{1}{1+f}\right) 
 \stackrel{f = \frac{\taudrift}{\tauevap}}{=} 
 \Qo(0) + \Qin\cdot \left( \frac{1}{1+ \tauevap/\taudrift} + \frac{\tauzero}{\tauevap} \frac{1}{1 + \taudrift/\tauevap}\right).
\end{equation}

\noindent
Bei Integration nur bis $t'$ kommt der Term $-\tauzero(1-\exp(-t'/\tauzero)$ hinzu:
\begin{equation}
 \Qin = Q_1(0) + Q_0(0) \frac{\tauzero}{\tauevap}\left[ 1 - \frac{\tauzero}{t'}(1 - \exp(-t'/\tauzero)) \right]
\end{equation}

\noindent
Gesamtladung:
\begin{equation}
 \Qtot = \Qo(0) + Q_1(0) + Q_0(0) \frac{\tauzero}{\tauevap}\left[ 1 - \frac{\tauzero}{t'}(1 - \exp(-t'/\tauzero)) \right]
\end{equation}

\noindent
Wenn man nun wieder annimmt, dass $Q_1(0)$ vernachlaessigbar ist und $Q_0(0) \approx \Qin$, folgt
\begin{equation}
 \boxed{\Qtot = \Qo(0) +  \Qin(0) \frac{\tauzero}{\tauevap}\left[ 1 - \frac{\tauzero}{t'}(1 - \exp(-t'/\tauzero)) \right]}.
\end{equation}

\noindent
Ohne diese Vereinfachung:
\begin{equation}
 \Qtot = \Qo(0) +  \Qin(0) \cdot \left( \frac{1}{1+ \tauevap/\taudrift} + \frac{\tauzero}{\tauevap} \frac{1}{1 + \taudrift/\tauevap}\right) \cdot 
 \frac{\tauzero}{\tauevap} \left[ 1 - \frac{\tauzero}{t'}(1 - \exp(-t'/\tauzero)) \right].
\end{equation}

\subsubsection{Fuer $\taudrift \lesssim \tauevap$}
Wenn nun $\taudrift$ nicht $\ll \tauevap$, aber weiterhin $\taudrift < \tauevap$ und nur $\taurec$ vernachlaessigt wird, erhalte ich 
\begin{equation}
  \boxed{f = \frac{\taudrift}{\tauevap - \taudrift}} \,\,\,\textrm{und } \,\,\, \boxed{\frac{1}{\tauzeroprime} = \frac{1}{\tauzero} - \frac{f}{\taub} \approx \frac{1}{\tauevap} + \frac{1}{\taurec}}
\end{equation}
\noindent
Vernachlaessigt man $\frac{f}{\taub}$ macht man bei 100 K einen Fehler im Promillbereich, bei 150\,K wieder ca 10\,\%.
Dann lauten $n_1$ und $\ndrift$:

\begin{equation}
 n_1 = \frac{\taudrift}{\tauevap - \taudrift} n_0(0) \exp{(-t / \tauzero)}. 
\end{equation}

\noindent
und $\ndrift$, wieder mittels $\frac{\dd \ndrift}{\dd t} = + \frac{n_1}{\taudrift}$:

\begin{equation}
 \ndrift = \frac{\tauzero}{\tauevap - \taudrift} n_0(0) (1 - \exp{(-t / \tauzero)}). 
\end{equation}

\noindent
Zur Diskussion 7.5.1 aendert sich also nur der Faktor $\tauzero / \tauevap$ zu $\tauzero / (\tauevap - \taudrift)$. 

\subsection{Fuer $\tauevap \approx \taudrift$ und $\tauevap < \taudrift$}
Dafuer suche ich noch entsprechende Loesungen fuer $f$. 
So sollte $Q=Q(E)$ bei fester Temperatur am besten direkt aus $f$ folgen. 

\subsection{Resuemee}
Das DGL-System wird allgemein geloest mit der Annahme, dass das System hinreichend schnell gegen ein Gleichgewicht mit konstanten Verhaeltnis $n_1/n_0$ tendiert. 
Au3erdem wird angenommen, dass sich $\taub$ anfangs sehr schnell aendert, und dann gegen einen ungefaehr konstanten Wert strebt und somit nicht zeitabhaengig ist. 
(Diese Annahme wird von Ihnen auch implizit getaetigt). 
Es muessen insbesondere a priori keine Annahmen ueber die relative Groe3en von Zeitkonstanten und Dichten getroffen werden, um das DGL-System zu loesen. 

Es gibt eine Gleichung fuer $f$, die fuer verschiedene Faelle ausgewertet wird, und nicht verschiedene DGL-Systeme die jeweils neu geloest werden muessten. 
Der Faktor $f$ stellt sich den Stoerungen entsprechend ein. 
Ohne Stoerung und bei hypothetischer gleicher Stoerung ($\taudrift = \taurec$) ergibt sich wie erwartet das BM-QGGW. 
Kleine Aenderungen ($\delta$) ergeben sinnvolle Loesungen. 
Im Experiment sind die Stoerungen oft gro3, da $\taudrift$ klein, dadurch ist $f$ fuer $T < $ 150\,K klein, und EQ (43) wurde entsprechend vereinfacht und die Loesungen diskutiert. 
Es folgt im einfachsten Fall EQ (60) und (67). 

Im Prinzip muesste ein quadratisches DGL-System geloest werden. 
Da dann aber die Zeitabhaengigkeit von $\taub$ fest im System steckt, kann $n_1/n_0 = f = $ konstant nicht mehr angenommen werden. 
Fuer dieses DGL-System habe ich noch keinen Loesungsansatz parat. 





