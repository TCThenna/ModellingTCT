
We have interpreted transient current measurements, also known as time-of-flight measurements, 
 performed at temperatures from 2\,K to 295\,K and for electric fields from $\SI{0.08}{\volt/\um}$ (in most cases) to $\SI{1.8}{\volt/\um}$
 by a model where electrons and holes during relaxation from their highly excited states down to their respective band edges are driven in opposite directions. 
Thereby, three ranges are created in the direction of the electric field, two outer ranges containing electrons or holes alone and a central inner range which contains 
overlapping concentrations of electrons and holes. 
In the latter range, electron-hole drops (EHD) form with a binding energy of 135\,meV compared to free carriers. 
The binding energy is lowered at increasing field strength, and this dependence has been quantitatively modelled by considering the asymmetric potential of free excitons
 (which are constitutive states in the formation of EHDs) in an electric field. 
EHDs provide a current only at elevated temperatures upon thermal activation and dissociation into free electrons and holes causing the characteristic temperature-dependence of the current pulse profiles observed. 
Full agreement is found between the model prediction and the total measured charge at the collecting electrode as a function of temperature. 
Finally, we have observed maxima in the transit time only for electrons as a function of temperature prominent at low electric fields. 
This effect could be traced back to the thermal re-population between fast and slow electrons in the upper and lower valley-orbit states induced by the electric field. 
This explanation appears to be convincing as it models all observed characteristic features though quantitative agreement could not been achieved.

Our interpretation is a surprising manifestation that excitons and electron-hole drops often regarded as rather academic can play a vital role in elementary realistic low-to-room temperature measurements in diamond.
