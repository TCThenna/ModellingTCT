
In high-energy physics, semiconductors are used as solid-state ionisation chambers for charged particle detection profiting from their versatility. 
Besides well-established silicon-based sensors, diamond detectors are an emergent technology due to their exceptional electrical properties~\cite{book:PhysApplCVD}. 
Diamond is a wide band gap semiconductor and in the field of high-energy physics its usage growths steadily due to its radiation hardness and usability in harsh environments. 
Remarkable properties as high breakdown fields and low leakage currents are a direct result from its wide band gap of ($\SI{5.47}{\eV}$)~\cite{1964} 
 rendering very precise signal current measurements possible. 

Excitonic properties in various semiconductors are usually studied using optical methods detecting the photon emitted during the exciton's radiative decay. 
Therefore, those excitons that thermally dissociate elude optical detection. 
In this work, the experimental set-up resembles a charged particle detection experiment. 
We therefore measure the current induced from drifting charge carriers in an externally applied electric field, i.e.\ the thermally dissociated excitons,
 which represent the complementary part to those radiatively recombined excitons. 
To this end, we apply the transient current technique (TCT), which has frequently been used for the determination of the charge carrier drift
 time in different materials \cite{TCT,Eremin1996388,pernegger:073704,PSSA:PSSA200561929,Fink2006227,PSSA:PSSA200561915}.

We discuss previously reported data \cite{jansen:173706} under new aspects including an excitonic model describing the temperature and electric field dependence of the measured transients. 
In order to explain the measurements in a wide range of electric field strengths and sample temperature our model includes the interplay of electron-hole-droplets (EHD), free excitons
 and free charge carriers, which are interconnected by various time constants. 
 
The paper is structured as follows:
We present the experimental technique in section~2 and the measured current profiles as a function of the temperature and the electric field strength in section~3. 
These current profiles are further analysed in terms of collected charge, shape and drift time. 
Section~4 discusses a physical model describing the data. 